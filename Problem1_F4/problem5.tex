\documentclass[a4paper, 11pt]{article}
\usepackage{comment} 
\usepackage{fullpage} 
\usepackage{hyperref}
\usepackage{amsmath}

\usepackage{booktabs}

\usepackage[ruled]{algorithm2e} 
\renewcommand{\algorithmcfname}{ALGORITHM}

\begin{document}

\noindent
\large\textbf{PROBLEM 5 } \hfill \textbf{Dhruviben Modi} \\
\normalsize SOEN 6011 \hfill \textbf{40166396} \\
 SOFTWARE ENGINEERING PROCESSES \hfill Due Date: August 5, 2022 \\
\hfill Github address : git@github.com:mdhruvi/SOEN-6011-project.git
\section{Unit test cases}

\subsection{Test Case 1}
\begin{itemize}
    \item \textbf{ID} : TC1
    \item \textbf{Requirement ID} : FR1, FR2, FR4
    \item \textbf{Name of Test Case} : testGammafunctionforPositiveInteger()
    \item \textbf{Test Priority} : High
    \item \textbf{Pre-condition} : Start the programme, then type a value into the console.
    \item \textbf{Test Data} : 142, 234
    \item \textbf{Expected Outcome} : 1.8981437590761713E243, $\infty$
    \item \textbf{Actual Outcome} : 1.8981437590761713E243, $\infty$
    \item \textbf{Test Result} : Success
\end{itemize}

\subsection{Test Case 2}
\begin{itemize}
    \item \textbf{ID} : TC2
    \item \textbf{Requirement ID} : FR2, FR5
    \item \textbf{Name of Test Case} : testGammafunctionforDecimalNumber()
    \item \textbf{Test Priority} : High
    \item \textbf{Pre-condition} : Start the programme, then type a value into the console.
    \item \textbf{Test Data} : 21.14
    \item \textbf{Expected Outcome} : 3.699869886541974E18
    \item \textbf{Actual Outcome} : 3.699869886541974E18
    \item \textbf{Test Result} : Success
\end{itemize}

\subsection{Test Case 3}
\begin{itemize}
    \item \textbf{ID} : TC3
    \item \textbf{Requirement ID} : FR3
    \item \textbf{Name of Test Case} : testGammafunctionforPositiveInteger()
    \item \textbf{Test Priority} : High
    \item \textbf{Pre-condition} : Start the programme, then type a value into the console.
    \item \textbf{Test Data} : 0, -12
    \item \textbf{Expected Outcome} : Exception, Exception
    \item \textbf{Actual Outcome} : Exception, Exception
    \item \textbf{Test Result} : Success
\end{itemize}

\subsection{Test Case 4}
\begin{itemize}
    \item \textbf{ID} : TC4
    \item \textbf{Requirement ID} : FR3
    \item \textbf{Name of Test Case} : testPositiveLimitForGammaFunction()
    \item \textbf{Test Priority} : Medium
    \item \textbf{Pre-condition} : Start the programme, then type a value into the console.
    \item \textbf{Test Data} : 175
    \item \textbf{Expected Outcome} : $+\infty$
    \item \textbf{Actual Outcome} : Exception, Exception
    \item \textbf{Test Result} : Success
\end{itemize}

\subsection{Test Case 5}
\begin{itemize}
    \item \textbf{ID} : TC5
    \item \textbf{Requirement ID} : FR1
    \item \textbf{Name of Test Case} : testMainmethod()
    \item \textbf{Test Priority} : High
    \item \textbf{Pre-condition} : Start the programme, then type a value into the console.
    \item \textbf{Test Data} : 9 y ghjk 0 -4 200 67 y 150 0
    \item \textbf{Expected Outcome} : 40320.0, Exception, Exception, Error message, Error message, $\infty$, 5.443449390774432E92, Exception, 3.8089226376305687E260, Error message
    \item \textbf{Actual Outcome} : 40320.0, Exception, Exception, Error message, Error message, $\infty$, 5.443449390774432E92, Exception, 3.8089226376305687E260, Error message
    \item \textbf{Test Result} : Success
\end{itemize}


\subsection{Conclusion}
The Junit testing with the Junit 5 standard is used to test the Gamma Function programme. The running result and the expected outcome are compared using a series of "assertion" methods to determine whether all test cases either succeed or fail.


\end{document}
